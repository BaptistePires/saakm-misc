\section{\ipa architecture}
\par Dans cette section nous allons voir les détails de l'architecture d'\ipa.

\subsection{Les variables globales d'\ipa}
Dans \ipa, il y a deux variables globales importantes : \texttt{struct list\_head ipanema\_policies} et \texttt{per\_cpu(struct task\_struct *, ipanema\_current)}. Nous allons voir comment ces deux variables sont utilisées pour comprendre leur utilitée.\newline

\par{\textbf{ipanema\_policies}} La variables \texttt{struct list\_head ipanema\_policies} est une list chaînée qui contient toutes les politiques d'ordonnancement disponibles. Cela correspond à tous les modules qui ont été insérés en tant que politiques. Cette list est protégée par un \texttt{rwlock\_t ipanema\_rwlock} qui permet de la modifier/parcourir en toute sécurité.\newline
Losrque l'on souhaite ajouter une politique, on fait appel à la fonction \texttt{ipanema\_add\_policy(struct ipanema\_policy *)} qui va insérer la nouvelle politique dans la liste.\newline
Nous avons maintenant inséré une politique, mais aucun thread ne s'en sert sert pour l'instant. Pour ce faire, il va falloir se servir de l'appel système \texttt{SYS\_sched\_setattr} (voir \texttt{tools/ipanema/ipastart.c} pour un exemple d'utilisation). En somme, il appelle la fonction \texttt{\_\_sched\_setscheduler} qui va, si la politique demandée existe, modifier le champ de la \texttt{task\_struct.ipanema.policy} pour le faire pointer vers la politique souhaitée. Ensuite, cette liste sera par exemple utilisée dès que l'on a besoin de trouver une tâche à exécuter.\newline

\par{\textbf{ipanema\_current}} Cette variable est l'une des plus importantes. Elle est définie \texttt{per\_cpu}, ce qui signifie que pour tous les CPUs, il y a une variable d'allouée. Ces variables vont stocker le processus en cours d'exécution sur le coeur auquel elles appartiennent. Pour voir cela, il faut regarder la fonction \texttt{change\_state} qui gère toutes les transitions d'états d'\ipa. On peut voir que pour la transition \texttt{x -> RUNNING} (une tâche va commencer son exécution) : \texttt{per\_cpu(ipanema\_current, next\_cpu) = p} où \texttt{next\_cpu} est le CPU sur lequel la tâche va s'exécuter et \texttt{p} ladite tâche.\newline
Il faut noter que juste avant, la fonction \texttt{change\_rq} est appelée et la tâche \texttt{p} aura été retirée de sa runqueue où elle attendait à l'état \texttt{IPANEMA\_READY}.

