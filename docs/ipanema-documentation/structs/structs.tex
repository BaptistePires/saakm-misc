\section{Structures}
Dans cette section nous allons voir les différentes structures utilisées dans \ipa et leurs utilités

\subsection{\texttt{struct sched\_\ipam\_entity}}
Nous allons commencer par la structure qui représente une entité d'ordonnancement. La structure est déclarée de cette façon :\newline
\begin{lstlisting}[style=CStyle]
struct sched_ipanema_entity {
	union {
		struct rb_node node_runqueue;
		struct list_head node_list;
	};

	/* used for load balancing in policies */
	struct list_head ipa_tasks;

	int just_yielded;
	int nopreempt;

	enum ipanema_state state;
	struct ipanema_rq *rq;

	/* Policy-specific metadata */
	void *policy_metadata;

	struct ipanema_policy *policy;
};
\end{lstlisting}

\par Les deux premiers champs, \texttt{struct rb\_node node\_runqueue} et \texttt{struct list\_head node\_list} sont utilisés pour maintenir une tâche dans une runqueue. Dû à l'union, on peut être dans une seule runqueue à un moment donner (soit dans une runqueue qui se sert d'un arbre rouge-noir comme CFS, ou une runqueue qui utilise une liste comme un algorithme FIFO par exemple).\\

\par Le champ \texttt{struct list\_head ipa\_tasks} est utilisé par les différentes politiques d'ordonnancement pendant l'équilibrage de charge. Par exemple, si on migre des tâches d'un coeur \textit{$C_1$} à un coeur \textit{$C_2$}, on a besoin de retirer les tâches de la runqueue de \textit{$C_1$} en ayant le verrou sur celle-ci. Quand on retire les tâches on peut les ajouter à la liste \texttt{ipa\_tasks}. Une fois qu'on a retirée toutes les tâches qu'on souhaite migrer, on prend le verrour sur la runqueue de \textit{$C_2$} et on peut itérer sur la liste que l'on vient de construire.\\

\par \texttt{int just\_yield} permet de savoir si on vient d'appeler \texttt{sched\_yield()} ou non.\\
\par \texttt{int nopreempt}

\par \texttt{enum ipanema\_state state } représente l'état d'une entitté \ipa. Les états possibles sont les suivants :
\begin{lstlisting}[style=CStyle]
enum ipanema_state {
	IPANEMA_NOT_QUEUED, /* La tache n'est pas dans une runqueue */
	IPANEMA_MIGRATING, /* La tache est en train de migrer d'un coeur a un autre */
	IPANEMA_RUNNING, /* La tache est dans une runqueue et s'execute */
	IPANEMA_READY, /* La tache est dans une runqueue mais ne s'execute pas */
	IPANEMA_BLOCKED, /* La tache est bloque (I/O, lock, etc...) */
	IPANEMA_TERMINATED, /* La tache meurt (?) */
	/*
	 * A special state that is equivalent to IPANEMA_READY, but makes it
	 * possible to figure out that the state change came from tick().
	 */
	IPANEMA_READY_TICK
};
\end{lstlisting}

\par \texttt{struct ipanema\_rq *rq} c'est un pointeur vers la \texttt{struct ipanema\_rq} sur laquelle la tâche s'exécute (ou va s'exécuter).

\subsubsection{\texttt{struct ipanema rq}}
Now we will take a look at the \texttt{struct ipanema\_rq}. It's the structure that represents a runqueue in \ipa, fairly simple struct made of the following fields :

\begin{lstlisting}[style=CStyle]
struct ipanema_rq {
	enum ipanema_rq_type type;
	union {
		struct rb_root root;
		struct list_head head;
	};

	/*
	 * cpu to which this runqueue belongs (for per_cpu runqueues).
	 * For runqueues that are not per_cpu, rq->cpu > NR_CPUS.
	*/
	unsigned int cpu;
	enum ipanema_state state;

	/* Number of tasks in this runqueue*/
	unsigned int nr_tasks;

	int (*order_fn)(struct task_struct *a, struct task_struct *b);
};
\end{lstlisting}

\par \texttt{enum ipanema\_rq\_type} defines what kind of structure the runqueue will use. The enum is defined as follow : \texttt{enum ipanema\_rq\_type \{ RBTREE, LIST, FIFO \};}. As we can see, three types of runqueues are supported now. One with an red-black tree structure, one with a list and with a fifo list.
\par The union of \texttt{struct rb\_root} and \texttt{struct list\_head} is the root of either the tree or list of tasks in this runqueue.
\par \texttt{int cpu} is the id of the cpu this runqueue belongs to.
\par \texttt{enum ipanema\_state} demander à Redha pour être sûr. Defines the state in which the tasks must be when added to this runqueue.
\par \texttt{unsigned int nr\_tasks} The number of tasks currently in the runqueue.
\par \texttt{int (*order\_fn)(struct task\_struct *a, struct task\_struct *b)} is function pointer that will be called when a task is added to the runqueue or whenever an ordering is needed. It is not mandatory for the FIFO runqueues.


\par That concludes the part for the runqueue.