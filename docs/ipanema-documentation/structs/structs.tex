\section{Structures}
Dans cette section nous allons voir les différentes structures utilisées dans \ipa et leurs utilités

\subsection{\texttt{struct sched\_\ipam\_entity}}
Nous allons commencer par la structure qui représente une entité d'ordonnancement. La structure est déclarée de cette façon :\newline
\begin{lstlisting}[style=CStyle]
struct sched_ipanema_entity {
	union {
		struct rb_node node_runqueue;
		struct list_head node_list;
	};

	/* used for load balancing in policies */
	struct list_head ipa_tasks;

	int just_yielded;
	int nopreempt;

	enum ipanema_state state;
	struct ipanema_rq *rq;

	/* Policy-specific metadata */
	void *policy_metadata;

	struct ipanema_policy *policy;
};
\end{lstlisting}

\par Les deux premiers champs, \texttt{struct rb\_node node\_runqueue} et \texttt{struct list\_head node\_list} sont utilisés pour maintenir une tâche dans une runqueue. Dû à l'union, on peut être dans une seule runqueue à un moment donner (soit dans une runqueue qui se sert d'un arbre rouge-noir comme CFS, ou une runqueue qui utilise une liste comme un algorithme FIFO par exemple).\\

\par Le champ \texttt{struct list\_head ipa\_tasks} est utilisé par les différentes politiques d'ordonnancement pendant l'équilibrage de charge. Par exemple, si on migre des tâches d'un coeur \textit{$C_1$} à un coeur \textit{$C_2$}, on a besoin de retirer les tâches de la runqueue de \textit{$C_1$} en ayant le verrou sur celle-ci. Quand on retire les tâches on peut les ajouter à la liste \texttt{ipa\_tasks}. Une fois qu'on a retirée toutes les tâches qu'on souhaite migrer, on prend le verrour sur la runqueue de \textit{$C_2$} et on peut itérer sur la liste que l'on vient de construire.\\

\par \texttt{int just\_yield} permet de savoir si on vient d'appeler \texttt{sched\_yield()} ou non.\\
\par \texttt{int nopreempt}

\par \texttt{enum ipanema\_state state } représente l'état d'une entitté \ipa. Les états possibles sont les suivants :
\begin{lstlisting}[style=CStyle]
enum ipanema_state {
	IPANEMA_NOT_QUEUED, /* La tache n'est pas dans une runqueue */
	IPANEMA_MIGRATING, /* La tache est en train de migrer d'un coeur a un autre */
	IPANEMA_RUNNING, /* La tache est dans une runqueue et s'execute */
	IPANEMA_READY, /* La tache est dans une runqueue mais ne s'execute pas */
	IPANEMA_BLOCKED, /* La tache est bloque (I/O, lock, etc...) */
	IPANEMA_TERMINATED, /* La tache meurt (?) */
	/*
	 * A special state that is equivalent to IPANEMA_READY, but makes it
	 * possible to figure out that the state change came from tick().
	 */
	IPANEMA_READY_TICK
};
\end{lstlisting}

\par \texttt{struct ipanema\_rq *rq} c'est un pointeur vers la \texttt{struct ipanema\_rq} sur laquelle la tâche s'exécute (ou va s'exécuter).