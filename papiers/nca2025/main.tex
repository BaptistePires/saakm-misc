\documentclass[conference]{IEEEtran}
\IEEEoverridecommandlockouts
% The preceding line is only needed to identify funding in the first footnote. If that is unneeded, please comment it out.
\usepackage{cite}
\usepackage{amsmath,amssymb,amsfonts}
\usepackage{algorithmic}
\usepackage{graphicx}
\usepackage{textcomp}
\usepackage{xcolor}


% Commands
\newcommand{\note}[1]{\textcolor{red}{\textbf{note:}\textit{[#1]}}}
\newcommand{\todo}[1]{\textcolor{blue}{\textit{\textbf{todo:}[#1]}}}

\bibliography{biblio.bib}


\def\BibTeX{{\rm B\kern-.05em{\sc i\kern-.025em b}\kern-.08em
    T\kern-.1667em\lower.7ex\hbox{E}\kern-.125emX}}
\begin{document}

\title{SaaKM : Schedulers as Kernel Module*\\
{\footnotesize \textsuperscript{*}Note: Sub-titles are not captured in Xplore and
should not be used}
\thanks{Identify applicable funding agency here. If none, delete this.}
}

\author{\IEEEauthorblockN{1\textsuperscript{st} Baptiste Pires}
\IEEEauthorblockA{\textit{LIP6} \\
\textit{Sorbonne Université}\\
Paris France \\
baptiste.pires@lip6.fr}
\and
\IEEEauthorblockN{2\textsuperscript{nd} Given Name Surname}
\IEEEauthorblockA{\textit{dept. name of organization (of Aff.)} \\
\textit{name of organization (of Aff.)}\\
City, Country \\
email address}
\and
\IEEEauthorblockN{3\textsuperscript{rd} Given Name Surname}
\IEEEauthorblockA{\textit{dept. name of organization (of Aff.)} \\
\textit{name of organization (of Aff.)}\\
City, Country \\
email address}
\and
\IEEEauthorblockN{4\textsuperscript{th} Given Name Surname}
\IEEEauthorblockA{\textit{dept. name of organization (of Aff.)} \\
\textit{name of organization (of Aff.)}\\
City, Country \\
email address}
}

\maketitle

\begin{abstract}
\todo{abstract}
\end{abstract}

\begin{IEEEkeywords}
kernel, scheduler, module
\end{IEEEkeywords}

\section{Introduction}

\par CPU scheduling is a fundamental aspect of operating systems and plays a critical part regarding performances and security. With the increasing complexity of hardware and the diversity of workloads servers have to run, writing schedulers is tedious task. Linux provide a general purpose scheduler, Earliest Eligible Virtual Deadline First (EEVDF)\cite{eevdf} that is designed to handle those constraints. However, due to its genericity, it is not tailored to specific workloads and offers worse performance than application-targeted scheduling algorithms\cite{ghost,shenango}. \newline

\par Writing a Linux scheduler is complex task. On top of the performances constraints, it interracts with many parts of the kernel, requiring a deep understanding of the kernel APIs. Most schedulers are executed on multicore hardware, requiring to handle concurrency, synchronization and understanding the multiple contexts in which it can executes (interrupts, preempption). Those constraints create barrier for developers to modify or write schedulers. Furthermore, if you want to tweak your Linux scheduler, you have to recompile and redeploy the whole kernel which can be time consuming. Writing a scheduler with error will most likely end up in a kernel crash or causing data corruption, which can be quite tricky to debug and identify.\note{reformuler} \newline

\par This is why there are efforts to provide kernel developers easier, faster and safer way to write scheduling policies. There are two main approach to do so. ghOSt\cite{ghost} and Skyloft\cite{skyloft} both allow users to write schedulers that are executed in userspace. They do so by exposing the necessary scheduling data to userspace and implementing communications channels between the two. The second approach are schedulers executed in the kernel. sched\_ext\cite{schedext} was merged in the v6.12 of the Linux kernel and provide a way to to write schedulers as eBPF\cite{ebpf} programs. Enoki\cite{enoki} leverage the Linux kernel modules to write scheduling policies in Rust modules that are then loaded into the kernel.\newline

\par All of the current solutions require users to learn either a new language like Rust or a new subsystem like eBPF that can be seen as a constraint by developers. Our solution relies solely on linux kernel modules and the C langage which kernel developers are already familiar with. Our primary goal is to make writing schedulers easier and accessible to more developers, not to significantly increase performances.\newline

\par In this paper, we first present the state of the art for writing schedulers. Then we present the design and implementation of SaaKM, our contribution as an new way to write scheduling policies as Linux kernel modules. Finally, we compare SaaKM to sched\_ext to evaluate its performances and usability.\newline

% \par The introduction of sched\_ext\cite{schedext} in version v6.12 of the kernel aims to lower those barriers by providing a API to write schedulers using eBPF\cite{ebpf}. They do so by providing a set of helpers and hooks\note{pas sûr} to write scheduling policies that are injected at runtime and can be loaded/unloaded without the need to recompile and redploy the kernel.

% If you want to tweak it to better fit your workloads, you need to have a deep understanding of the core scheduler. Futhermore, the process to update, compile, deploy and test the new version is time consuming. That is why there are efforts to provide kernel developers ways to write, test and deploy schedulers faster. \newline


\section{Background}

\par The scheduler is a critical part of an operating system (OS) that manages the processes executions. It is required to take into account the hardware it runs on and it might need to adapt to a lot of different workloads. It is also tighly coupled with other parts of the OS like the memory management subsystem (especially regarding NUMA architectures), it therefore require a lot of knowledge to write and maintain a scheduler. When it comes to the Linux kernel, the Earliest Eligible Virtual Deadline First (EEVDF) scheduler is the default general purpose scheduler since v6.6. It is composed of 7000+ LoC alone and depends on a significant amount of C headers an4d kenel APIs. Because of its genericity it implements a lot of heuristics 
that can be hard to isolate and understand. It also causes a loss in performances in order to adapt to various workloads. That is why there is an effort in the industry and academia to offer kernel developpers a way to write and test schedulers easily\note{ghost,schedext,plugsched}. This need motivated and allowed the successful merge in the kernel of \textit{sched\_ext} in the Linux version v6.12. \\
In this section, we will go through the current options that are available to kernel developer to write and test schedulers in the linux kernel. First, we will see the \textit{sched\_class} API of the Linux kernel, then we will take a look at Google proposition, ghOSt\cite{ghost-paper} that delegate scheduling decisions to user space. Then, we will see \textit{plugsched} a solution proposed by Alibaba, finally we will see \textit{sched\_ext} the latest solution proposed by Meta that has been merged into the Linux kernel.\note{ajouter un passage sur la liste des sched class}

\paragraph{The \textit{sched\_class} API}
\todo{tableau comme CSD avec +/- de chaque solutions ?}


\par Linux scheduler subsystem offers an API to write scheduler through its \textit{sched\_class} structure. It contains a set of function pointers that has to be implemented by the schedulers and we will be called in the right place by the scheduler core. You still need to modify some datastructures of the kernel like the \textit{task\_struct} (it represents a task) and the \textit{rq} (it represents a runqueue) to store data of your scheduler. This already represents a lot of work because you need to understand how the core of the scheduler (6000+ LoC) works, when and how to take locks and all of the other synchronization mecanisms used (RCU, memory syncrhonizations, ...). \\ Once you have implemented yout scheduler, you need to modify the linking file \textit{vmlinux.lds.h} to add your brand new scheduling class. This is what will allow the kernel to find it and register it with the others. The last step is to compile and boot your kernel, and only then, you will be able to test and debug it. \\ 
In short, writing a scheduler using the \textit{sched\_class} API is a long and tedious process that requires a wide and deep knowledge of the Linux kernel. \note{réecrire + restructurer un peu cette partie avec schedma sched class + avantages et inconvénients}

\paragraph{ghOSt - Userspace scheduling}

\par In order to match the evolution of their needs, Google needed a way to write, test and deploy schedulers easily. That is the main motivation behind ghOSt\cite{ghost-paper}. It delegates scheduling decisions to userspace and allows to deploy new schedulers or features without having to recompile and deploy kernel images. ghOst architecture is devided into two parts. The first one lies in the kernel, using a \textit{sched\_class} to implement the core of ghOSt and to define a rich API exposed to the userspace. The second part is executed in userspace and is composed of \textit{agents} which are scheduling policies. The two parts communicate through \textit{message queues} that allows the kernel to notify the agents of a thread state change, a scheduling tick. The agents now have all the information needed to make scheduling decisions and can commit \textit{transactions} via shared memory. \todo{presentation resultats + limitations}

\paragraph{plugsched - Live scheduler update}
\par Plugsched takes another approach to give users a way to tweak and modify the scheduler. It leverage the modularization of the Linux kernel by isolating the scheduler code into a \textit{Linux kernel module} that users can then modify, recompile and insert it into a running kernel. To do so, they rewrite the code sections of threads to redirect modified functions. \note{on est forcés de de passer tous les threads dans la nouvelle politique}

\paragraph{sched\_ext}
\par \textit{sched\_ext} is a proposition by Meta that has been merged into the Linux kernel mainline in v6.12. It offers a way to write schedulers as \textit{eBPF} programs. They rely on the \textit{eBPF} verifier to provide safety and security. The eBPF API allows to communicate with userspace program quite easily through \textit{maps}. In this case, it is used to delegate some scheduling work to the userspace, such as in the scheduler \textit{scx\_rusty} where they offload all of the load balancing logic to a userspace program written in Rust. This solution comes with limitations as you need to learn the eBPF API (which is still less time consuming than the core of the Linux scheduler) and you can be limited by the eBPF verifier.


\par As we have seen in this section, the need for a way to write, test and easily deploy new schedulers in the Linux kernel is a pressing issue. Each of the presented solutions have their own advantages and drawbacks

\section{Design}
\label{sec:scheduler-as-a-kernel-module}

% \par We present \textit{Scheduler as a Kernel Module} (SaaKM), a framework that allows kernel developers to write schedulers as Linux kernel modules. Our main goal is to provide a way to write, test and deploy schedulers easily. To do so, we hide the complexity of the core scheduler (synchronization mecanisms, complex API) behind a set of functions corresponding to scheduling events that each scheduling policy must implement. We are also capable to have multiple policies loaded at the same time, allowing each applications to chose the scheduler that best fits its needs.

% \subsection{Design}

% \par In this section we present the disgn behind SaaKM,
\par In this section we present the design of SaaKM, our framework for writing schedulers as Linux kernel modules. First we present the goals of our approach, then we detail the design architeture.\newline

\subsection{Design goals} Writing schedulers in Linux is a tedious task so one of our main goals is to provide a simpler way to do that. The need to have a deep and wide knowledge of the Linux kernel can be a barrier also so we want to hide as much complexity as possible. Finally, testing and debugging a scheduler is also a criteria.\newline

\textbf{Ease of Use} An ideal solution would not require kernel developers to master a new langage nor a new subsystem of the kernel as it would defeat one the purpose of the solution, that is the accessibility part. By the nature of kernel development, they must already be knowledgeable in C and some of the kernel APIs and subsystems. Linux kernel modules answer perfectly to this need.

\textbf{Performance} The solution must incur the lowest overhead possible. As the scheduler is a critiacal part regarding performances, even a small overhead can cause significant performances degradation. We must keep our solution as ligthweiht as possible, giving the users the choice to add features that can cause this overhead if they want to.\newline

\textbf{Testing and Debugging} The current process of debuging and testing a scheduler class require to recompile and redeploy your kernel image each time you make a modification. This can be quite time consuming and slow down the development process. That is why our solution must not be quick and easy to test without the need reboot your machine.\newline
\subsection{Design overview}
\par SaaKM is an API to write schedulers as Linux kernel modules. It exposes a minimal set of exported functions to the modules that can be used to allocate, manage and destroy scheduling policies and data. Figure \ref{fig:sched-class-saakm} shows how SaAKM integrates itself with the current scheduling classes. We position ourselves right before the idle class to not disrupt higher priority classes that other threads runs.\newline

\par All of the SaaKM design relies on a structure composed of handlers that each policy must implement. Each handler maps to a scheduling event. Events are devided into two categories. Thread events correspond to all scheduling events related to a thread (e.g threak is created, waking up, ...). On the other hand, core events are related to the CPU cores (e.g scheduling tick, core becoming idle, ...). This distinction makes it clear for the user to know what it should do, hiding the complexity of the scheduling class API where a single function can be called from multiple paths. For instance, the enqueue\_task function of the scheduling class can be called from numerous places and for different reasons. It can be for a newly forked task, a migrating task, a waking up task. Our goal is to isolated those cases to let the user know what is actually happening without having to how it got called. Table \ref{tab:saakm-callbacks} shows an excerpt of those events.\newline

\par Furthermore, this division allows us to hide the complexity of the syucrhonization mecanisms of the core scheduler. As multicore and NUMA architectures are the quite common now, the scheduler must synchroniza data across cores quite often as two core may access the same data. It protects these data through the usage of locks, RCU and low level synchronization mecanisms like memory barriers. Thanks to our design, the user does not need to worry about these and assume that it has the right locks on the data whenever it is needed.\note{réecrire cette partie} \newline

\par To increase the flexibility of our solution, we support to have multiple policies loaded at the same time. This allows applications to have the right scheduler tailored for their needs.\note{completer}

\begin{figure}[htbp]
        \centering
        \includesvg[width=0.45\textwidth]{figures/linux-saakm-class.svg}
        \caption{SaaKM scheduling class with three policies loaded (refaire pour rendre plus visible)}
        \label{fig:linux-saakm-sched-class}
\end{figure}

\subsection{Implementation}
\par We implement SaaKM on Linux v6.12, the latest longterm release. In this section we will go through the implementation details.

\par We implement a minimal (less than 1500 LoC) scheduling class that implements all of the scheduling functions required by the core scheduler. We place ourselves between the ext and idle scheduling classes as shown in Figure \ref{fig:linux-saakm-sched-class} where there are three policies loaded. We define a structure \texttt{saakm\_module\_routines} that is composed of all the handler a policy must implement. Table \ref{tab:saakm-callbacks} shows an excerpt of these handlers.

\par Each policy must allocate and populate a strcture with its own handlers and they will be called by our scheduling class at the right places. Each handlers has at least one argument that is a pointer to its corresponding policy structure. Then depending on the event type of the handler (thread or core), it will have either a pointer to a structure \texttt{process\_event} or \texttt{core\_event} that contains the data related to the event.\newline

\par \textbf{Runqueues management} We provide a minimal runqueue definition that policies uses and includes in their own runqueues as a structure field. This allows to delegate the insertion/deletion and mandatory runqueue metadata fields to be handled by SaaKM. We provide three runqueue types. A FIFO, LIST and RED-BLACK tree. You must provide an ordering function for each type of runqueue, except for the FIFO that will always insert the task at the end of the queue. We provide these three types as they are the most common ones but thanks to its design it can be extended. Policies are required to initialize their runqueues before registering themselves.\newline 

\par \textbf{Thread states} We define our own thread states to make to not interfer with the core scheduler. We define 7 states represented in Figure \ref{fig:saakm-states}. We have one special state, SAAKM\_READY\_TICK that allows policy to notify SaaKM that it needs to trigger a reschedule. With this states, we are able to check for wrongful transitions and provide debug feedback to the user at runtime. This ease the debugging and testing process. We provide a function \texttt{change\_state(task, next\_state, next\_cpu, next\_rq)} that is the core of SaaKM threads states managmenent Policies must call it anytime they want to alter the state of a thread giving it the right arguments. It is responsible of threads states management. It calls the core schedulers functions to keep the kernel related data correct. It also add and remove threads from the runqueues and check that the operation is valid (i.e. the correct locks are held). Threads states transitions checks is done in this function.
\note{peut etre faire un diagramme d'etat qui montre la machine à etat et que c'est ça qui nous permet de detecter les erreurs ? mieux que tableau ?}\newline

\par \textbf{Thread Migration} To handle thread migration that occurs because of affinity change or a \texttt{exec} rebalance (a newly created thread is being run on a diffrent CPU than its parent), we introduce a new flag, OUSTED. This flags allows us to know that we are being dequeued becuase of those reason so we can simulate a fast block/unblock sequence to remove the thread from its current runqueue and insert it into the new one. \newline

\par \textbf{Load balancing} A central and critical part of scheduling is the load balancing. It consist of trying to distribute the load accross CPU cores according to some metrics. To do so, you may need to access data belonging to other cores. Locks are therefore needed to ensure the correctness of the data. We identified two types of load balancing. Periodic load balancing occurs at a fixed time interval and idle balancing. We define a new software interrupt, SCHED\_SOFTIRQ\_IPANEMA with a handler that calls policies \texttt{balancing\_select} function. Policies can them chose at which time interval they want to do load balancing. We also raise our interrupt when the fair scheduling kicks a CPU to do the NOHZ balancing.\note{est ce qu'on doit garder cette feature avec EXT ?} \\
Policies should also be able to do load balacing when a CPU is becoming idle. Whenever that happens, we call the policy handler to let it perform it. \newline

\par \textbf{Hardware topology} To allow policies to properly make load balancing, we export a per-cpu variable, \texttt{topology\_levels} that represent the underlying hardware toplogy. To do so, we rely on the already existing scheduling domains that are build at initialization of the scheduler subsystem. Policy can then take educated decisions based on the underlying hardware.\newline
% \par Futhermore, this division allows us to hide the complex synchronization mecanisms of the core scheduler. For instance, there is a rule that if you need to lock multiple runqueue, on top of the overhead it might causes, you must always lock them in the correct order (i.e. the runqueue whose CPU id is the lowest must always be locked first

\par \textbf{Multi-policy support} SaaKM supports to have multiple policies loaded at the same time. When it registers, it is given an unique incremental id that is used by application to select their policy. We then maintain them in a linked list ordered by insertion order. When a new thread must be elected, we go through the list of policies starting from the head of the list. The first policy to return a non NULL thread stops the search and the thread is elected. For load balancing, we also go through the full list to make sure that all policies can perform their load balance.
% \begin{table}[htbp]
%         % \centering
%         \caption{SaaKM thread states}
%         \begin{tabular}{|l|l|}
%         % \toprule
%         \hline
%         \textbf{State} & \textbf{Meaning} \\
%         % \midrule
%                 \hline
%                 \texttt{SAAKM\_NOT\_QUEUED} & Not in a runqueue \\
%                 \hline
%                 \texttt{SAAKM\_MIGRATING} & Being migrated \\
%                 \hline
%                 \texttt{SAAKM\_RUNNING} & Running on a CPU \\
%                 \hline
%                 \texttt{SAAKM\_READY} & Ready to run and in a runqueue \\
%                 \hline
%                 \texttt{SAAKM\_READY\_TICK} & Became ready from a \textit{tick} event \\
%                 \hline
%                 \texttt{SAAKM\_BLOCKED} & Blocked and cannot run \\
%                 \hline
%                 \texttt{SAAKM\_TERMINATED} & Dead \\
%                 \hline
%         \end{tabular}
        
% \label{tab:saakm-states}
% \end{table}
\begin{figure}[htbp]
        \centering
        \includesvg[width=0.5\textwidth]{figures/saakm-states.svg}
        \caption{SaaKM states finite state machine}
        \label{fig:saakm-states}
\end{figure}



% \par We implement a minimal (less than 1500 LoC) scheduling class that implements all of the scheduling functions requiered by the core scheduler. We place ourselves after the ext policy and before the idle one to not disturb higher priority scheduling classes (e.g RR, fair). We expose an API for the Linux kernel Modules (LKM) to enable them to register as scheduling policies. We store the policies in a linked list sorted by insertion date. Each policy is assigned a id that is unique and used by applications to select the policy they want to use. Figure ? shows the general architecture of SaaKM. \newline

\subsection{SaaKM workflow}

\par In this section we will go through SaaKM workflow ....\newline

\par \textbf{Registering a policy} To register a policy, during its initialization, a module must allocate and fill a \texttt{saakm\_policy} that contains all of the policy metadata (i.e. name, routines, id and metadata). It must also allocate then initialize its runqueues with the \texttt{saakm\_runqueue\_init} specifying the type of the runqueue (FIFO, LIST, RED-BLACK TREE) and an ordering function if needed. It can then register itself by calling the \texttt{saakm\_add\_policy}. \newline

\par \textbf{Selecting a policy} Selecting a policy is pretty straighforward. An application must define an \texttt{sched\_attr} structure and set the \texttt{sched\_policy} field to SCHED\_SAAKM and set the \texttt{sched\_saakm\_policy} with the value of the policy id it wants. From this point, the thread \texttt{task\_struct.sched\_class} field is set to our scheduling policy and will therefore be scheduled by SaaKM. \newline

\par \textbf{Debugging and testing} Being a kernel module, policies can leverage the already existing kernel debugging tools such KGDB\cite{kgdb}, ftrace, the debugfs interface and many others. On top of that, we provide a kernel configuration options to check the transitions of the thread states and trigger panic when it happens to give user control and a way to inspect the kernel. We also exposes a procfs directory that list the current policies loaded with some metadata. Finally, we use the \texttt{sysfs} interface to turn on or off some debugging features : thread transitions checks, print in dmesg the wrongful transitions and print the poilcy
thread transitions checks, print the wrongful transitions and print the scheduling classes calls. \newline



% \par To register a policy, a module must implement a set of functions defined in the structure \texttt{saakm\_module\_routines}. It is composed of handlers that map to scheduling events. Table \ref{tab:saakm-callbacks} shows an excerpt of those events. Events are devided into two categories. Thread events consist of all events related to a thread (e.g thread is waking up, a new thread is created, ...) and core events that are related to core management (e.g core becoming idle, scheduling tick, ...). This distinction allow user to exactly know the path it is. For instance, take the \texttt{enqueue\_task} function from the \texttt{sched\_class} structure. X \todo{est-ce que c'est une bonne idée comme example ?} paths lead to its call. For instance, when a thread is waking up and need to be enqueued back on a runqueue or if a thread was just created. We hide this complexity behind the event and call the handlers at the right places. This way, the user does not need to differentiate paths, it just have to worry about the event currently happening. \note{ancienne version, bouger des phrases ailleurs}\newline
% \par \textit{Synchronizations} In order to hide the complexity of the synchronization mechanisms, we split the selection of a CPU and the actual enqueuing in two steps. The first one (\texttt{new\_prepare} and \texttt{unblock\_prepare}) must returns the id of the CPU on which the thread should run. Then comes the second step (\textit{new\_place} and \texttt{unblock\_place}) where the actual enqueuing takes place. It frees the user from knowing which locks are currently held and which ones it is allowed to take. It is needed for instance when a thread is created. \note{completer et réécrire} \newline
% \par \textit{Thread states} We overload thread states with ours to make it easier for the user to manage\note{?}. As we have the full control over our states, this allows us to check for wrongful transitions and provide debug feedback to the users. This ease the process of development and debugging. \newline
% \par \textit{Runqueues mamagement} Each policy must allocate its runqueues ...

% \par \textit{Thread migration} To handle thread migration, we introduce a new flag, OUSTED. This allows to discriminate between a dequeue coming from a blocking event from a migration. We then simulate a block/unblock sequence to let the policies remove the thread from its runqueue and insert it into the new one.

% \note{parler des etats des threads (et verifications via la machine à état), de comment on passe des données via les structures *\_event, peut etre un exemple du workflow (insertion usage etc), parler de la gestion des runqueues, la migration avec le flag OUSTED, gestion du load balancing,}

% \par We rely on the \textit{sched\_class} API presented above to implement a minimal core scheduler (less than 1500 LOC) that will only be used to register policies and call callbacks at the right places. To not interfer with the existing schedulers such as EEVDF or sched\_ext, we place our scheduling class right before the idle one.\\ Each policy must implement a set of functions that map to specific scheduling events (e.g wakeup, scheduling tick, ...) and then register itself to our scheduling class. To not interfer with the existing schedulers such as EEVDF or sched\_ext, we place our scheduling class right before the idle one. Figure ? shows the general architecture of SaaKM\\
% \par{} There are two types of events. The \textit{thread events} and \textit{core events}. Table \ref{tab:saakm-callbacks} shows some examples of these events and their descriptions. We devided the selection of a CPU and the actual enqueuing in the runqueue in two steps. This is needed when a task is newly created task is woken up for the first time. In a first time, it locks the task and calls \texttt{new\_prepare} which returns the CPU. At this point, our scheduler sees the task for the first time, it needs to initialize and allocated metadata for it. Only then, the runqueue of the CPU is locked and \texttt{new\_place} is called to actually enqueue the task. Thanks to this architecture, we are able to hide the complexity of the syncrhonization mechanisms to the user.

% \begin{table}[htbp]
% \caption{Table Type Styles}
% \begin{center}
% \begin{tabular}{|c|c|c|c|}
% \hline
% \textbf{Table}&\multicolumn{3}{|c|}{\textbf{Table Column Head}} \\
% \cline{2-4} 
% \textbf{Head} & \textbf{\textit{Table column subhead}}& \textbf{\textit{Subhead}}& \textbf{\textit{Subhead}} \\
% \hline
% copy& More table copy$^{\mathrm{a}}$& &  \\
% \hline
% \multicolumn{4}{l}{$^{\mathrm{a}}$Sample of a Table footnote.}
% \end{tabular}
% \label{tab1}
% \end{center}
% \end{table}
\begin{table}[htbp]
        % \centering
        \caption{Extract of \texttt{saakm\_module\_routines} functions}
        \begin{tabular}{|l|l|}
        % \toprule
        \hline
        \textbf{Function} & \textbf{Description} \\
        % \midrule
        \hline
                \textbf{Thread events} &\\
                % \hline
                new\_prepare(p) & Return CPU id where \texttt{p} should run\\
                new\_place(task, core) & Insert \texttt{p} into \texttt{core} runqueue\\
                new\_end(task, core) & Insert \texttt{p} into \texttt{core} runqueue\\
                unblock\_place(task) & Return CPU id where \texttt{p} should run\\
                unblock\_prepare(task) & Insert \texttt{p} into \texttt{core} runqueue\\
                block(task) & \texttt{p} is blocking\\
                yield(task) & \texttt{p} is yielding the CPU\\
                tick(task) & Called at each scheduler tick \\
                terminate(task) & \texttt{task} dies \\
                \hline
                \textbf{Core events} & \\
                % \hline
                schedule(core) & Called when a task must be elected\\
                newly\_idle(core) & Called right after \texttt{schedule}\\& if no READY found\\
                enter\_idle(core) & No task to run\\
                exit\_idle(core) & A task is runnable\\
                balancing\_select(core) & Called to do load balancing\\
                core\_entry(core) & \texttt{core} is becoming online\\
                core\_exit(core) & \texttt{core} is going offline\\
                % \midrule
        % \bottomrule
                \hline 

                \textbf{Policy management} & \\
                % \hline
                init() & Called to initialize policy\\
                free\_metadata() & Free policy metadata\\
                checkparam\_attr(sched\_attr) & Check validity of sched\_attr\\
                setparam\_attr(sched\_attr) & Use sched\_attr to set values\\
                setparam\_attr(sched\_attr) & Use sched\_attr to set values\\
                \hline
        \end{tabular}
        
\label{tab:saakm-callbacks}
\end{table}

% \begin{itemize}
        % \item[-]{Task events} are all events related to task, such as : a blocking task, a task waking up.
        % \item[-]{Core events} are all events related to the CPUs, such as : a scheduling tick, a CPU becoming idle.
% \end{itemize}
% \par 
% In order to hide the complexity of the locks mecanisms and make it invisible to the user, we have to make the selection of a CPU and the actual enqueuing in two step. This is needed when a task \textit{p} is for waking up on CPU_0 but the scheduler decides that it should now run on CPU_1. Firstly, we need to lock CPU_0 to completly remove \textit{p} before we can enqueue it on CPU_1. \note{reecrire}

\section{Evaluation}

\par In this section we present our experimental setup

\subsection{Experimental Setup}
\par We run our benchmarks on a server running Debien 12 bookworm with a our patched Linux kernel v6.12. The server is equiped with 10 core and 20 SMT threads (Intel(R) Core(TM) i9-10900K CPU @ 3.70GHz), 20MB of L3 cache and 64GO of RAM. We run each benchmark at least 50 times and present the mean and standard deviation. \\
We compare ourselves to the ext scheduler. We implement a minimal FIFO scheduler for comparaison. Functions to select CPUs are based on threads PIDs as e want to mesaure the overhead of SaaKM and not the efficiency of a scheduling algorithm. \\
We ran a total of X applications, using the phoronix test suite and builtin benchmarks from applications. \newpage

\subsection{Evaluation goal}
\par The goal of our experimentation is to evaluate if SaaKM is a viable solution for writing schedulers and how it performs compared to existing solution. We are motivated by the

\begin{itemize}
        \item Definir les métriques
        \item Definir ce à quoi on se compare
        \item Présenter l'env et les benchmarks et les motiver
        \item Présentation et analyse des résultats
\end{itemize}


% \section{Ease of Use}

% \subsection{Maintaining the Integrity of the Specifications}

% The IEEEtran class file is used to format your paper and style the text. All margins, 
% column widths, line spaces, and text fonts are prescribed; please do not 
% alter them. You may note peculiarities. For example, the head margin
% measures proportionately more than is customary. This measurement 
% and others are deliberate, using specifications that anticipate your paper 
% as one part of the entire proceedings, and not as an independent document. 
% Please do not revise any of the current designations.

% \section{Prepare Your Paper Before Styling}
% Before you begin to format your paper, first write and save the content as a 
% separate text file. Complete all content and organizational editing before 
% formatting. Please note sections \ref{AA}--\ref{SCM} below for more information on 
% proofreading, spelling and grammar.

% Keep your text and graphic files separate until after the text has been 
% formatted and styled. Do not number text heads---{\LaTeX} will do that 
% for you.

% \subsection{Abbreviations and Acronyms}\label{AA}
% Define abbreviations and acronyms the first time they are used in the text, 
% even after they have been defined in the abstract. Abbreviations such as 
% IEEE, SI, MKS, CGS, ac, dc, and rms do not have to be defined. Do not use 
% abbreviations in the title or heads unless they are unavoidable.

% \subsection{Units}
% \begin{itemize}
% \item Use either SI (MKS) or CGS as primary units. (SI units are encouraged.) English units may be used as secondary units (in parentheses). An exception would be the use of English units as identifiers in trade, such as ``3.5-inch disk drive''.
% \item Avoid combining SI and CGS units, such as current in amperes and magnetic field in oersteds. This often leads to confusion because equations do not balance dimensionally. If you must use mixed units, clearly state the units for each quantity that you use in an equation.
% \item Do not mix complete spellings and abbreviations of units: ``Wb/m\textsuperscript{2}'' or ``webers per square meter'', not ``webers/m\textsuperscript{2}''. Spell out units when they appear in text: ``. . . a few henries'', not ``. . . a few H''.
% \item Use a zero before decimal points: ``0.25'', not ``.25''. Use ``cm\textsuperscript{3}'', not ``cc''.)
% \end{itemize}

% \subsection{Equations}
% Number equations consecutively. To make your 
% equations more compact, you may use the solidus (~/~), the exp function, or 
% appropriate exponents. Italicize Roman symbols for quantities and variables, 
% but not Greek symbols. Use a long dash rather than a hyphen for a minus 
% sign. Punctuate equations with commas or periods when they are part of a 
% sentence, as in:
% \begin{equation}
% a+b=\gamma\label{eq}
% \end{equation}

% Be sure that the 
% symbols in your equation have been defined before or immediately following 
% the equation. Use ``\eqref{eq}'', not ``Eq.~\eqref{eq}'' or ``equation \eqref{eq}'', except at 
% the beginning of a sentence: ``Equation \eqref{eq} is . . .''

% \subsection{\LaTeX-Specific Advice}

% Please use ``soft'' (e.g., \verb|\eqref{Eq}|) cross references instead
% of ``hard'' references (e.g., \verb|(1)|). That will make it possible
% to combine sections, add equations, or change the order of figures or
% citations without having to go through the file line by line.

% Please don't use the \verb|{eqnarray}| equation environment. Use
% \verb|{align}| or \verb|{IEEEeqnarray}| instead. The \verb|{eqnarray}|
% environment leaves unsightly spaces around relation symbols.

% Please note that the \verb|{subequations}| environment in {\LaTeX}
% will increment the main equation counter even when there are no
% equation numbers displayed. If you forget that, you might write an
% article in which the equation numbers skip from (17) to (20), causing
% the copy editors to wonder if you've discovered a new method of
% counting.

% {\BibTeX} does not work by magic. It doesn't get the bibliographic
% data from thin air but from .bib files. If you use {\BibTeX} to produce a
% bibliography you must send the .bib files. 

% {\LaTeX} can't read your mind. If you assign the same label to a
% subsubsection and a table, you might find that Table I has been cross
% referenced as Table IV-B3. 

% {\LaTeX} does not have precognitive abilities. If you put a
% \verb|\label| command before the command that updates the counter it's
% supposed to be using, the label will pick up the last counter to be
% cross referenced instead. In particular, a \verb|\label| command
% should not go before the caption of a figure or a table.

% Do not use \verb|\nonumber| inside the \verb|{array}| environment. It
% will not stop equation numbers inside \verb|{array}| (there won't be
% any anyway) and it might stop a wanted equation number in the
% surrounding equation.

% \subsection{Some Common Mistakes}\label{SCM}
% \begin{itemize}
% \item The word ``data'' is plural, not singular.
% \item The subscript for the permeability of vacuum $\mu_{0}$, and other common scientific constants, is zero with subscript formatting, not a lowercase letter ``o''.
% \item In American English, commas, semicolons, periods, question and exclamation marks are located within quotation marks only when a complete thought or name is cited, such as a title or full quotation. When quotation marks are used, instead of a bold or italic typeface, to highlight a word or phrase, punctuation should appear outside of the quotation marks. A parenthetical phrase or statement at the end of a sentence is punctuated outside of the closing parenthesis (like this). (A parenthetical sentence is punctuated within the parentheses.)
% \item A graph within a graph is an ``inset'', not an ``insert''. The word alternatively is preferred to the word ``alternately'' (unless you really mean something that alternates).
% \item Do not use the word ``essentially'' to mean ``approximately'' or ``effectively''.
% \item In your paper title, if the words ``that uses'' can accurately replace the word ``using'', capitalize the ``u''; if not, keep using lower-cased.
% \item Be aware of the different meanings of the homophones ``affect'' and ``effect'', ``complement'' and ``compliment'', ``discreet'' and ``discrete'', ``principal'' and ``principle''.
% \item Do not confuse ``imply'' and ``infer''.
% \item The prefix ``non'' is not a word; it should be joined to the word it modifies, usually without a hyphen.
% \item There is no period after the ``et'' in the Latin abbreviation ``et al.''.
% \item The abbreviation ``i.e.'' means ``that is'', and the abbreviation ``e.g.'' means ``for example''.
% \end{itemize}
% An excellent style manual for science writers is \cite{b7}.

% \subsection{Authors and Affiliations}
% \textbf{The class file is designed for, but not limited to, six authors.} A 
% minimum of one author is required for all conference articles. Author names 
% should be listed starting from left to right and then moving down to the 
% next line. This is the author sequence that will be used in future citations 
% and by indexing services. Names should not be listed in columns nor group by 
% affiliation. Please keep your affiliations as succinct as possible (for 
% example, do not differentiate among departments of the same organization).

% \subsection{Identify the Headings}
% Headings, or heads, are organizational devices that guide the reader through 
% your paper. There are two types: component heads and text heads.

% Component heads identify the different components of your paper and are not 
% topically subordinate to each other. Examples include Acknowledgments and 
% References and, for these, the correct style to use is ``Heading 5''. Use 
% ``figure caption'' for your Figure captions, and ``table head'' for your 
% table title. Run-in heads, such as ``Abstract'', will require you to apply a 
% style (in this case, italic) in addition to the style provided by the drop 
% down menu to differentiate the head from the text.

% Text heads organize the topics on a relational, hierarchical basis. For 
% example, the paper title is the primary text head because all subsequent 
% material relates and elaborates on this one topic. If there are two or more 
% sub-topics, the next level head (uppercase Roman numerals) should be used 
% and, conversely, if there are not at least two sub-topics, then no subheads 
% should be introduced.

% \subsection{Figures and Tables}
% \paragraph{Positioning Figures and Tables} Place figures and tables at the top and 
% bottom of columns. Avoid placing them in the middle of columns. Large 
% figures and tables may span across both columns. Figure captions should be 
% below the figures; table heads should appear above the tables. Insert 
% figures and tables after they are cited in the text. Use the abbreviation 
% ``Fig.~\ref{fig}'', even at the beginning of a sentence.

% \begin{table}[htbp]
% \caption{Table Type Styles}
% \begin{center}
% \begin{tabular}{|c|c|c|c|}
% \hline
% \textbf{Table}&\multicolumn{3}{|c|}{\textbf{Table Column Head}} \\
% \cline{2-4} 
% \textbf{Head} & \textbf{\textit{Table column subhead}}& \textbf{\textit{Subhead}}& \textbf{\textit{Subhead}} \\
% \hline
% copy& More table copy$^{\mathrm{a}}$& &  \\
% \hline
% \multicolumn{4}{l}{$^{\mathrm{a}}$Sample of a Table footnote.}
% \end{tabular}
% \label{tab1}
% \end{center}
% \end{table}

% Figure Labels: Use 8 point Times New Roman for Figure labels. Use words 
% rather than symbols or abbreviations when writing Figure axis labels to 
% avoid confusing the reader. As an example, write the quantity 
% ``Magnetization'', or ``Magnetization, M'', not just ``M''. If including 
% units in the label, present them within parentheses. Do not label axes only 
% with units. In the example, write ``Magnetization (A/m)'' or ``Magnetization 
% \{A[m(1)]\}'', not just ``A/m''. Do not label axes with a ratio of 
% quantities and units. For example, write ``Temperature (K)'', not 
% ``Temperature/K''.

% \section*{Acknowledgment}

% The preferred spelling of the word ``acknowledgment'' in America is without 
% an ``e'' after the ``g''. Avoid the stilted expression ``one of us (R. B. 
% G.) thanks $\ldots$''. Instead, try ``R. B. G. thanks$\ldots$''. Put sponsor 
% acknowledgments in the unnumbered footnote on the first page.

% \section*{References}

% Please number citations consecutively within brackets \cite{b1}. The 
% sentence punctuation follows the bracket \cite{b2}. Refer simply to the reference 
% number, as in \cite{b3}---do not use ``Ref. \cite{b3}'' or ``reference \cite{b3}'' except at 
% the beginning of a sentence: ``Reference \cite{b3} was the first $\ldots$''

% Number footnotes separately in superscripts. Place the actual footnote at 
% the bottom of the column in which it was cited. Do not put footnotes in the 
% abstract or reference list. Use letters for table footnotes.

% Unless there are six authors or more give all authors' names; do not use 
% ``et al.''. Papers that have not been published, even if they have been 
% submitted for publication, should be cited as ``unpublished'' \cite{b4}. Papers 
% that have been accepted for publication should be cited as ``in press'' \cite{b5}. 
% Capitalize only the first word in a paper title, except for proper nouns and 
% element symbols.

% For papers published in translation journals, please give the English 
% citation first, followed by the original foreign-language citation \cite{b6}.

% % \printbibliography
% \begin{thebibliography}{00}
% \bibitem{b1} G. Eason, B. Noble, and I. N. Sneddon, ``On certain integrals of Lipschitz-Hankel type involving products of Bessel functions,'' Phil. Trans. Roy. Soc. London, vol. A247, pp. 529--551, April 1955.
% \bibitem{b2} J. Clerk Maxwell, A Treatise on Electricity and Magnetism, 3rd ed., vol. 2. Oxford: Clarendon, 1892, pp.68--73.
% \bibitem{b3} I. S. Jacobs and C. P. Bean, ``Fine particles, thin films and exchange anisotropy,'' in Magnetism, vol. III, G. T. Rado and H. Suhl, Eds. New York: Academic, 1963, pp. 271--350.
% \bibitem{b4} K. Elissa, ``Title of paper if known,'' unpublished.
% \bibitem{b5} R. Nicole, ``Title of paper with only first word capitalized,'' J. Name Stand. Abbrev., in press.
% \bibitem{b6} Y. Yorozu, M. Hirano, K. Oka, and Y. Tagawa, ``Electron spectroscopy studies on magneto-optical media and plastic substrate interface,'' IEEE Transl. J. Magn. Japan, vol. 2, pp. 740--741, August 1987 [Digests 9th Annual Conf. Magnetics Japan, p. 301, 1982].
% \bibitem{b7} M. Young, The Technical Writer's Handbook. Mill Valley, CA: University Science, 1989.
% \end{thebibliography}
% \vspace{12pt}
% \color{red}
% IEEE conference templates contain guidance text for composing and formatting conference papers. Please ensure that all template text is removed from your conference paper prior to submission to the conference. Failure to remove the template text from your paper may result in your paper not being published.

\end{document}
